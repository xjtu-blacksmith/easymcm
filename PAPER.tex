%%%%%%%%%%%%%%%%%%%%%%%%%%%%%%
% 	   美赛模板,正文部分		 
%          PAPER.tex         
%%%%%%%%%%%%%%%%%%%%%%%%%%%%%%

\documentclass[12pt]{article}

% 请在此填写控制号、题号和标题,年份不需要填(自动以当前电脑时间年份为准)
\usepackage[1234567]{easymcm}\problem{A}   
\usepackage{palatino} % 这个是COMAP官方杂志采用的字体,如不需要可注释掉,以使用默认字体
\title{An MCM Paper Made by Team 1234567}  % 标题

% 如您参加的是ICM(即选择了D/E/F题),请使用以下的命令修改Summary Sheet题头
% \renewcommand{\contest}{Interdisciplinary Contest in Modeling (ICM) Summary Sheet}

% 正文开始
\begin{document}
%%%%%%%%%%%%%%%%%%%%%%%%%%%%%%%%%%%%%%%%%
%%            请在此填写摘要            %%
%% 请勿编译/排版此文件,请编译PAPER.tex!  %%
%%%%%%%%%%%%%%%%%%%%%%%%%%%%%%%%%%%%%%%%%
\begin{abstract}\small
    Here is the abstract of your paper.

    Firstly, that is ...

    Secondly, that is ...

    Finally, that is ...

    % 美赛论文中无需注明关键字。若您一定要使用,
    % 请将以下两行的注释号'%'去除,以使其生效;
    % 若您不使用,可直接将这段注释删除
    % \vspace{5pt}
    % \textbf{Keywords}: MATLAB, mathematics, LaTeX.

\end{abstract}




%%%%%%%%%%%%%%%%%%%%%%%%%%%%%%%%%%%%%%%%%%
% 如不理解以下部分中各命令的含义,请勿修改! %
%%%%%%%%%%%%%%%%%%%%%%%%%%%%%%%%%%%%%%%%%%

%---------以下生成sheet页----------
% 下面的语句可调整全文行距为标准值的0.6倍,请自行使用
% \renewcommand{\baselinestretch}{0.6}\normalsize
\maketitle  			% 生成sheet页
\thispagestyle{empty}   % 不要页眉页脚和页码
\setcounter{page}{-100} % 此命令仅是为了避免页码重复报错,不要在意

%---------以下生成目录----------
\newpage
\tableofcontents
\thispagestyle{empty}   % 不要页眉页脚和页码
\newpage

%---------以下生成正文----------
\setlength\parskip{0.8\baselineskip}  % 调整段间距
\setcounter{page}{1}    % 从正文开始计页码
\pagestyle{fancy}		% 摘要请到ABSTRACT.tex中填写

\section{Introduction}
\subsection{Problem Background}
Here is the problem background ...

Two major problems are discussed in this paper, which are:
\begin{itemize}
    \item Doing the first thing.
    \item Doing the second thing.
\end{itemize}

\subsection{Literature Review}
A literatrue\cite{1} say something about this problem ...

\subsection{Our work}
We do such things ...

\begin{enumerate}[\bfseries 1.]
    \item We do ...
    \item We do ...
    \item We do ...
\end{enumerate}

\section{Preparation of the Models}
\subsection{Assumptions}

\subsection{Notations}
The primary notations used in this paper are listed in \textbf{Table \ref{tb:notation}}.
\begin{table}[!htbp]
\begin{center}
\caption{Notations}
\begin{tabular}{cl}
	\toprule
	\multicolumn{1}{m{3cm}}{\centering Symbol}
	&\multicolumn{1}{m{8cm}}{\centering Definition}\\
	\midrule
	$A$&the first one\\
	$b$&the second one\\
	$\alpha$ &the last one\\
	\bottomrule
\end{tabular}\label{tb:notation}
\end{center}
\end{table}

\section{The Models}
\subsection{Model 1}
\subsubsection{Detail 1 about Model 1}
\begin{equation}
    e^{i\theta}=\cos\theta+i\sin\theta.
\end{equation}

\section{Strengths and Weaknesses}
\subsection{Strengths}
\begin{itemize}
    \item First one...
    \item Second one ...
\end{itemize}

\subsection{Weaknesses}
\begin{itemize}
    \item Only one ...
 \end{itemize}

\begin{thebibliography}{99}
\addcontentsline{toc}{section}{References}  %引用部分标题("Refenrence")的重命名
\bibitem{1}Elisa T. Lee, Oscar T. Survival Analysis in Public Health Research. \emph{Go. College of Public Health}, 1997(18):105-134.
\bibitem{2}Wikipedia: Proportional hazards model. 2017.11.26. \texttt{\\https://en.wikipedia.org/wiki/Proportional\_{}hazards\_{}model}
\end{thebibliography}


% ==============以下为附录内容,如您的论文中不需要程序附录请自行删除====================
\clearpage
\begin{subappendices}						% 附录环境
\section*{Apendix: The source codes}		% 附录标题可以自行修改
\addcontentsline{toc}{section}{Appendix}  	% 将附录内容加入到目录中

This MATLAB program is used to calculate the value of variable $a$.
\begin{lstlisting}[language=Matlab, caption=\texttt{temp.m}]
a = 0;
for i = 1:5
	a = a + 1;
end
\end{lstlisting}

This LINGO program is used to search the optimize solution of 0-1 problem.
\begin{lstlisting}[language=Lingo, caption=\texttt{temp.lg4}]
model:
sets:
WP/1..12/: M, W, X;
endsets
data:
M = 2 5 18 3 2 5 10 4 11 7 14 6;
W = 5 10 13 4 3 11 13 10 8 16 7 4;
enddata
max = @sum(WP:W*X);
@sum(WP: M * X)<=46;
@for(WP: @bin(X));
end
\end{lstlisting}

\end{subappendices}
% =================================================================================



\end{document}